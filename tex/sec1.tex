\chapter{Introduction}\label{sec1}
\thispagestyle{empty}

% relevance
Supervisory agencies need to identify weaknesses of the institutions under supervision at an early stage to be able to react in a timely manner. The prevented collapse of the globally active and systematically important Swiss bank Credit Suisse by regulatory intervention in early 2023 have again demonstrated the necessity of banking supervision and the need for quick reactions to financial distressed banks to sustain financial and general economic stability. A few months prior to the 2023 banking crisis, openAI first launched ChatGPT, providing the general public access to their large language model (LLM) what drew huge public attention. As of August 2024, chatGPT has 200 million weekly users \citep{reuters2024}, indicating the huge potential of the application of LLMs, which potentially could improve the banking supervision process.  This process involves monitoring of a banks' risk management and controls, internal processes and procedures, governance and financial and operational soundness by reviewing banks' internal documents, discussion with banks' personnel and independent analysis \citep{hirtle2022}. In their work, supervisory agencies are often faced with resource constraints \citep{prenio2024}. Hence, supervisory agencies such as the Swiss Financial Market Supervision Authority (FINMA) adopt a risk-oriented supervisory approach, concentrating on institutions that demand increased attention according to their supervisory category and a dynamic rating process \citep{finma2024}. \\

% decision situation
To improve or extend this rating process, this study tries to leverage information published by news media providers. Given a set of news articles, a sentiment score is sought which captures the tone of the news media about the corresponding bank. Hence, the fast amount of text published by news media sources must be filtered for relevant information about individual banks, before the sentiment of this information can be classified as negative, neutral or positive. The goal is to construct an indicator which signals increased risk of the corresponding bank before traditional measures. This study hence contributes to the current initiatives by supervisory authorities in developing novel tools based on artificial intelligence and big data analytics to support their supervisory processes. Following \cite{broeders2018}, these novel methods are referred to as supervisory technology (suptech). As reported in \cite{prenio2024}, a survery conducted by the Financial Stability Institute (FSI) and the Bank of International Settlements (BIS) Innovation Hub in 2023 shows that 47 of the 50 surveyed supervisory agencies have ongoing suptech initiatives, whereby 44 of them have already deployed a suptech tool. \\

% literature
Quantifying news sentiment and using the resulting sentiment scores for prediction is inspired by the financial literature, which has shown several promising results. Hence, besides focusing on sentiment analysis for improving the investment decision-making process, there is a branch of the literature which focuses on financial distress prediction. The literature can be categorised to corporation-expressed, media-expressed and internet-expressed sentiment analysis. Traditional literature in sentiment analysis for financial research questions uses traditional methods such as bag-of-words models to classify the sentiment of text and hence does not leverage the possibilities provided by novel methods. Additionally, Swiss banks which are not classified as global systemically important are not typically covered in research studies. \\

% proposed solution approach/results
This study contributes to the media-expressed sentiment for financial distress literature by using a large language model for the classification of text and by testing the predictive performance of the retrieved sentiment indicator on data from 2022 until 2024 for a sample of Swiss and European banks. More specifically, we classify the sentiment using a Bidirectional Encoder Representations from Transformers (BERT) model developed by Google researchers in 2018 \citep{devlin2019}, which was further fine-tuned on financial texts. Due to its design, the model is able to understand words in context of the remaining text and hence is expected to perform better when assessing the sentiment of a text compared to traditional methods. We assess the predictive power of this sentiment indicator for common risk proxies to determine whether the sentiment indicator is able to forecast the riskiness of the corresponding bank. The results show, dependent on the sample and risk proxy, that the news sentiment indicator has predictive power. Some further modifications in the construction of the indicator and in the application for forecasting are necessary to implement it in the context of an early warning system. \\

% structure
This paper is structured as follows. First, we present the proposed procedure to obtain the sentiment scores in \mbox{Section~\ref{sec2}}. Next, in \mbox{Section~\ref{sec3}}, we provide an overview of the literature on the foundations of sentiment analysis in finance, the current research of sentiment analysis relevant in a supervisory context, and the methods used to measure the financial distress of banks. In \mbox{Section~\ref{sec4}}, we describe the methods used to construct the news sentiment indicator and discuss the benefits and limitations. We present the data used in this analysis, the experimental design as well as its results in \mbox{Section~\ref{sec5}}. Finally, we conclude in \mbox{Section~\ref{sec6}}.


\newpage
% -------------------------------------------


% relevanz des themas
%Supervisory agencies need to identify weaknesses of the institutions under supervision at an early stage to be able to react in a timely manner. The prevented collapse of the globally active and systematically important Swiss bank Credit Suisse by regulatory intervention in early 2023 have again demonstrated the necessity of banking supervision and the need for quick reactions to financial distressed banks to sustain financial and general economic stability. A few months prior to the 2023 banking crisis, openAI first launched ChatGPT, providing the general public access to their developed large language model (LLM), drawing huge public attention and has as of August 2024 200 million weekly users \citep{reuters2024}, indicating the huge potential of the application of LLMs, which potentially could improve the banking supervision process. \\

% entscheidungssituation
%Banking supervision includes monitoring of a banks' risk management and controls, internal processes and procedures, governance and financial and operational soundness by reviewing banks' internal documents, discussion with banks' personnel and independent analysis \citep{hirtle2022}. As stated in \cite{prenio2024}, supervisory agencies are often faced with limited resources. The Swiss Financial Market Supervision Authority (FINMA) is no exception, which is why it adopts a risk-oriented supervisory approach, concentrating on institutions that demand increased attention due to their supervisory category and a dynamic rating process \citep{finma2024}. Hence, the decision problem a supervisory agency like FINMA is confronted with, is the allocation of human resources on institutions to be supervised. \\

% stand der technik
%To overcome the resource constraints, supervisory agencies already invest in novel tools based on artificial intelligence (AI) and big data analytics to support their supervisory processes. Following \cite{broeders2018}, these novel methods are referred to as supervisory technology (suptech). As reported in \cite{prenio2024}, a survery conducted by the Financial Stability Institute (FSI) and the Bank of International Settlements (BIS) Innovation Hub in 2023 shows that 47 of the 50 surveyed supervisory agencies have ongoing suptech initiatives, whereby 44 of them have already deployed a suptech tool. An example of a tool discussed in \cite{prenio2024} which is already incorporated in the supervisory process of investment funds is an early warning system, which, among other functionalities, searches the web for relevant information about financial institutions that might signal potential supervisory issues. Two other examples of tools mentioned in \cite{prenio2024} are a tool that automates risk annotation for documents submitted by financial institutions and a social media monitoring tool, which aims to analyse public sentiment regarding financial institutions. These tools demonstrate the demand to supplement the existing supervisory process by methodologies to automatically analyse textual data as well as including information from alternative data sources. Both these tools are however not embedded effectively in a supervisors process. \\

% l�sungsansatz und resultate
%As the 2023 banking crisis and the investments in suptech from supervisory agencies stressed the importance of innovative methods of identifying risks in the banking sector and with the rise of LLMs yielding new possibilities, this analysis examines the questions whether LLMs can be applied for monitoring bank risk based on news media sentiment. For this, a news sentiment indicator is constructed using a Bidirectional Encoder Representations from Transformers (BERT) model developed by Google researchers in 2018 \citep{devlin2019}, which was further fine-tuned for financial text. Due to its design, the model is able to understand words in context of other words in a sentence and hence is expected to perform better when assessing the sentiment of a text compared to traditional methods. Then, the predictive power of this sentiment indicator for common risk proxies is assessed to determine whether the sentiment indicator is able to forecast the riskiness of the correspondent bank. The results show, that while the news sentiment indicator has some predictive power, some further modifications in constructing the indicator and using it for forecasting would be necessary to implement it in the context of an early warning system. \\


% organization ---
%This paper is structured as follows. First, the decision problem a supervisory agency is faced with in their day-to-day work processes is presented. Next, an overview of the literature on the foundations of sentiment analysis in finance, the current research of sentiment analysis relevant in a supervisory context, and the methods used to measure the financial distress of banks is given. In \mbox{Section~\ref{sec4}}, the methods used to construct the news sentiment indicator are described and its benefits and limitations are discussed. \mbox{Section~\ref{sec5}} presents the data used for this analysis, the experimental design and its results, followed by the conclusion in \mbox{Section~\ref{sec6}}.


\cleardoublepage
