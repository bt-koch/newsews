\chapter{Introduction}\label{sec1}
\thispagestyle{empty}

% relevanz des themas
The series of bank failures in the US banking sector as well as the prevented collapse of the globally active and systematically important Swiss bank Credit Suisse by regulatory interaction and the acquisition by UBS in early 2023 have demonstrated the importance of banking supervision to sustain financial and general economic stability. Supervisory agencies need to identify weaknesses of the institutions under supervision at an early stage to be able to react in a timely manner. However, responding in a timely manner is made more difficult by limited resources available to the authorities \citep{prenio2024}. Simultaneously, many authorities are complementing these investments by investing in novel tools based on innovative technology to support the supervisory work \citep{prenio2024}. Following \cite{dicastri2019}, the innovative technology refers to applications of artificial intelligence (AI) and big data analyis applied to support supervisory processes and is referred to as supervisory technology (suptech).  A few months prior to the 2023 banking crisis, openAI first launched ChatGPT, providing the general public access to their developed large language model (LLM), drawing huge public attention and has as of August 2024 200 million weekly users \citep{reuters2024}, indicating the huge potential of the application of LLMs. \\

% entscheidungssituation
Banking supervision includes monitoring of a banks' risk management and controls, internal processes and procedures, governance and financial and operational soundness by reviewing banks' internal documents, discussion with banks' personnel and independent analysis \cite{hirtle2022}. As stated in \cite{prenio2024}, supervisory agencies are often faced with limited resources. The Swiss Financial Market Supervision Authority (FINMA) is no exception of limited human resources, which is why it adopts a risk-oriented supervisory approach, concentrating on institutions that demand increased attention due to their supervisory category and a dynamic rating process.\footnote{\url{https://www.finma.ch/en/supervision/banks-and-securities-firms/}} Hence, the decision problem a supervisory agency like the FINMA is confronted with, is the allocation of human resources on institutions to be supervised. \\

% stand der technik
To overcome the resource constraints, supervisory agencies already invest in suptech to support their processes. As reported in \cite{prenio2024}, a survery conducted by the Financial Stability Institute (FSI) and the Bank of International Settlements (BIS) Innovation Hub in 2023 of 50 supervisory authorities shows that 47 authorities have ongoing suptech initiatives, whereby 44 of them have already deployed a suptech tool. An example of a tool discussed in \cite{prenio2024} which is already incorporated in the supervisory process of investment funds is an early warning system, which among other functionalities searches the web for relevant information about financial institutions that might signal potential supervisory issues. Two other examples of tools in�\cite{prenio2024} which are however not embedded effectively in the supervisory process is a tool which automates risk annotation of document which are submitted by financial institutions or a social media monitoring tool, which aims to examine public sentiment on financial institution. The main problem of the social media monitoring tool is, that some providers have made it impossible to obtain the required amount of data.�\\

% l�sungsansatz und resultate
Hence, as the 2023 banking crisis and the investments in suptech from supervisory agencies stress the importance of innovative methods of identifying risks in the banking sector and the rise of LLMs yield new possibilities, this analysis examines the questions whether LLMs can be applied for monitoring bank risk based on news media sentiment. For this, a news sentiment indicator is constructed using a Bidirectional encoder representations from transformers (BERT) model developed by Google researchers in 2018 \citep{devlin2019} and was specifically trained for financial text. Due to the design, the model is able to understand words in context of other words in a sentence and hence is expected to perform better when assessing the sentiment of a text compared to traditional methods. Then, the predictive power of this sentiment indicator for common risk proxies is assessed to determine whether the sentiment indicator is able to forecast the riskiness of the correspondent bank. The results show that the news sentiment indicator has some predictive power, some further modifications in constructing the indicator and using it for forecasting would be necessary however to implement it in the context of an early warning system. \\


% organization ---
This paper is structured as follows. First, the decision problem a supervisory agency is faced with in their day-to-day work processes. Next, we present the literature about the foundations of sentiment analytics in finance, the current research which is relevant in a supervisory context and about how financial distress of banks can be measured. We conclude by the contribution of this analysis to the literature. In section \ref{sec4}, we describe the methods used to construct the news sentiment indicator and its benefits and limitations. Section \ref{sec5} describes the data used for this analysis, the experimental design and its results before we conclude in section \ref{sec6}.


\cleardoublepage
