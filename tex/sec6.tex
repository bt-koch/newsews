\chapter{Conclusion and Outlook}\label{sec6}
\thispagestyle{empty}

This study tries to answer the question whether the sentiment of content published in news articles about individual banks can be quantified and whether the resulting metric has predictive power for the risk of the corresponding bank. The underlying issue is whether an indicator derived from news media articles can improve the decision-making process for banking supervisors by helping to determine whether a particular bank requires closer monitoring. The study shows that current developments in the research area of large language models can be applied to classify the sentiment of financial news articles. To assess the predictive power of the sentiment indicator, various models for different risk proxies such as Credit Default Swap spreads, the maximum drop of stock prices in a leading observation window or the stock price volatility are fitted and evaluated. The results suggest that there is predictive power depending on the news media data source, the sample and observation period as well as the risk proxy considered. One main advantage of the proposed indicator is that it enables to nowcast the level of financial distress of a bank in a high frequency. For this, it does not require that the corresponding bank is traded publicly and hence expands the group of banks for which the risk can be nowcasted already using market derived proxies. Furthermore, as seen in the literature as well as in some results in this study, sentiment indicators can be leading to market derived metrics, hence signaling increased risk earlier. To utilise this sentiment indicator as an early warning signal from a supervisory perspective, additional study and further extensions of the methods presented are necessary. A decision rule needs to be implemented which defines when additional supervisory activity is required due to the signals from the sentiment indicator. There should only be a signal if the negative sentiment does not represent a regular adverse event, but rather a threat to the financial health of a bank. Additionally, if textual data is expected to play a significant role in further optimising the supervisory process, it may be beneficial to train a large language model specifically for supervisory purposes, rather than relying on a model trained for general financial applications. The methods presented could further be extended to include additional textual data sources such as publications of the banks itself or social media content.


%\textit{conclusion}: \\
%summarize motivation, model(s) and their results including the limitations \\
%
%\noindent
%\textit{limitations}
%\begin{itemize}
%	\item still noise of my way to classify relevancy. e.g. also LLM could make this job?
%	\item relatively short obs period for cds (max 15months)
%	\item no threshold when critital for supervision / supervision probably won't care about normal stock price decrease or something. this would need further analysis
%	\item maybe look at spreads also between risk proxies and between sentiment indicator
%\end{itemize}
%
%analysis shows that there is some information contained in news media sentiment about swiss banks. it can indicate higher risk and might be used for supervision or also credit risk management. to act as an early warning, we would need some decision rule when we need to look more closely at a bank. \\
%
%no true cause and effect in this analysis, more a correlational study. 
%
%\noindent
%\textit{extensions}:
%\begin{itemize}
%	\item fine tune LLM fine tune an llm to classify whether a text is relevant for a specific context
%	\item ...
%	\item principal component analysis -> we could expect a large commonality and co-movement in the risk proxies, i guess this is also backed by theory/literature? maybe also in sentiment scores?
%	\item maybe training a llm for sentiment classification in the perspective of banking supervision would be good, which also can distinguish between more negative news i.e. can assign values smaller than -1
%	\item design a decision rule, e.g. classify banks on high risk or something
%	\item out of sample analysis
%\end{itemize}
%
%problem with incentives of news media -> "sell adds" ie make noisy articles \\
%
%maybe auch viele artikel auf plattformen wie refinitiv von maschine geschrieben -> reflektiert dies eventuell sowieso nur informationen welche bereits eingepreist sind? \\
%
%wichtig auch zu sagen dass high frequency indicators schwierig ist f�r individual banks, daher news sentiment o.�. attraktiv




\cleardoublepage
