\chapter{Literature Review}\label{sec3}
\thispagestyle{empty}


\section{Foundations of Sentiment Analysis in Finance}

While the application of sentiment analytics in banking supervision is a relatively new field, the foundations that financial markets react to news was already shown in \cite{debondt1985} and \cite{cutler1988}. Not only did the financial markets react, they tend to over- and underreact in an irrational way. The reaction is most pronounced for unexpected and dramatic news events. \cite{tetlock2007} was one of the first which constructed a quantitative indicator based on news article sentiment to analyse the interactions between news articles and the stock market using a bag-of-words algorithm. The results show that high media pessimism predicts falling stock prices. HIER NOCH WO DIE H�UFIGSTEN APPLICATIONS SIND MIT EINIGEN EXAMPLE PAPERS. \\

\section{Sentiment Analysis for Banking Supervision}

Hier noch irgend ein �bergang dass die methoden auch f�r banking supervision interessant sind. hier auch erw�hnen dass relativ neu ist und literatur noch "d�nn" ist. Following \cite{kearney2014}, we categorize the sentiment analysis for banking supervision literature according go their corresponding data source to corporation-expressed sentiment, media-expressed sentiment and internet-expressed sentiment. \\

Corporation-expressed sentiment analysis tries to capture the tone of the corporate executives regarding the performance, management and the strategy of a firm, utilizing various corporate disclosures. \cite{hajek2023} and \cite{huang2023} show that extending existing methods for general financial distress prediction methods of firms with sentiment measures from firms' annual reports improves the predictive performance. \cite{gandhi2019} focuses on the U.S. banking sector and suggests, that the early warning systems of banking supervision authorities which are based on financial statement data could be extended by an indicator derived from the frequency of negative words in the banks' annual reports. While most studies analyze written text published by the corresponding firm, \cite{hajek2023} show that the methods can also be applied to spoken text by successfully utilizing speech emotion recognition of earning conference call transcripts. \\
% more https://www.doi.org/10.1109/SSCI.2016.7850006, \cite{nopp2015}

 Focusing on content published by news agencies, media-expressed sentiment tries to capture the tone in a specific article. \cite{smales2016} analyses the media-expressed sentiment on banks. The results suggests that while CDS spreads as a marked determined measure reacts to news sentiment, LIBOR-OIS spreads as a bank determined measure does not. Furthermore, by analyzing the relationship between the sentiment of certain news topics and CDS spreads, the studies of \cite{roeder2020} suggest that further categorizing news sentiment into topics might improve the predictive power of the indicator. These studies show that negative sentiment in news sentiment could indicate higher risk of the corresponding bank. However, as \cite{agoraki2022} show, positive sentiment can also lead to increased financial stability by increased risk appetite by bank investors. \cite{borovkova2017} extend the sample to systemically important financial institutions and show that an indicator for systemic risk constructed from news sentiment leads other systemic risk measures by as long as 12 weeks. \\
% https://doi.org/10.21203/rs.3.rs-2305052/v1 (attention not peer reviewed)
 
 Internet-expressed sentiment tries to capture the tone of the discussion in social media posts which can be written by the bank itself, public or private institutions or private persons. \cite{fernandez2021} derive a sentiment indicator for the Mexican financial sector as a whole based on tweets, which captures sources of financial stress which are not reflected in quantitative risk measures and show that this indicator correlates with measures for financial risk. \cite{illia2021} analyse the relationship between the sentiment of tweets about a single bank and its daily business performance and find that tweet sentiment might affect bank performance when embedded in a larger conversation. The recent banking crisis in 2023 opened the discussion whether social media is a new factor which accelerates bank run behavior. Indeed first results in \cite{cookson2023} analysing the bank run on Silicon Valley Bank in 2023 indicate that social media sentiment could amplify bank run risk.
% \cite{mendoza2024}
% https://ieeexplore.ieee.org/stamp/stamp.jsp?arnumber=10648688
% https://www.sciencedirect.com/science/article/pii/S0378426620302314

\section{Metrics for Financial Distress of Banks}

HIER MUSS ICH NOCH AUSREICHEND ZEIT EINPLANEN!!!

%e.g. volatility or maximum drawdown
%
%gibt es eventuell kategorien?
%
%wie marketmetrics: cds, stock price (vola), mdd \\
%
%wie accounting data: liquid assets, ...

\section{Contribution to the Literature}

\newpage



bis hier sollte lit review kommen


\cleardoublepage
