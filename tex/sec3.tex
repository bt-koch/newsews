\chapter{Literature Review}\label{sec3}
\thispagestyle{empty}

%https://www.bis.org/publ/qtrpdf/r_qt2412b.pdf


\section{Foundations of Sentiment Analysis in Finance}

While the application of sentiment analysis in banking supervision is a relatively new field, \cite{debondt1985} and \cite{cutler1988} already showed that financial markets react to news. \cite{debondt1985} show that they do not only react but tend to overreact to unexpected and dramatic news events. \cite{tetlock2007} was one of the first which constructed a quantitative indicator based on news article sentiment to analyse the interactions between news articles and the stock market using a bag-of-words algorithm. The results show that high media pessimism predicts falling stock prices. \cite{du2024} show that most literature focuses on applications in predictive analytics for asset prices on the stock market, volatility for financial risk prediction, for portfolio selection, price movements in the foreign exchange market as well as price and volatility prediction in the cryptocurrency market.
% eventuell noch schiller und kahneman & tversky?

\section{Sentiment Analysis for Banking Supervision}

While research in financial sentiment analysis methods often focuses on the investment decision-making process, they may also have potential applicability in the context of banking supervision. The body of literature directly related to sentiment analysis in banking supervision remains limited. Hence, we also cover studies of sentiment analysis for general financial distress prediction, which could easily be applied to the banking sector specifically. Following \cite{kearney2014}, we categorise the sentiment analysis for banking supervision literature according to their corresponding data source to corporation-expressed sentiment, media-expressed sentiment and internet-expressed sentiment. \\

Corporation-expressed sentiment analysis tries to capture the tone of the corporate executives regarding the performance, management and the strategy of a firm based on various corporate disclosures. \cite{hajek2023} and \cite{huang2023} show that the predictive performance of general financial distress prediction methods for firms can be improved by extending them with sentiment measures which are based on firms' annual reports. By analysing the U.S. banking sector, \cite{gandhi2019} suggests that the early warning systems of banking supervision authorities which are based on financial statement data could be extended by an indicator derived from the frequency of negative words in the banks' annual reports. While these studies focus on classifying the sentiment of text, \cite{hajek2023} show that speech emotion recognition can be applied on transcripts of earning conference calls and show that managerial emotions recognition can improve financial distress prediction. \\
% more https://www.doi.org/10.1109/SSCI.2016.7850006, \cite{nopp2015}

 Focusing on content published by news agencies, media-expressed sentiment tries to capture the tone in a specific article. \cite{smales2016} analyses the media-expressed sentiment on banks. The results suggests that while Credit Default Swap (CDS) spreads as a marked determined measure reacts to news sentiment, the spread between the London Interbank Offered Rate (LIBOR) and Overnight Indexed Swaps (OIS) as a bank determined measure does not. Furthermore, by analysing the relationship between the sentiment of certain news topics and CDS spreads, the studies of \cite{roeder2020} suggest that further categorising news sentiment into topics might improve the predictive power of the indicator. These studies show that negative sentiment in news articles could indicate higher risk of the corresponding bank. However, as \cite{agoraki2022} note, positive sentiment can also lead to increased financial instability by increased risk appetite by bank investors. \cite{borovkova2017} extend the sample to systemically important financial institutions and show that an indicator for systemic risk constructed from news sentiment leads other systemic risk measures by as long as 12 weeks. \\
% https://doi.org/10.21203/rs.3.rs-2305052/v1 (attention not peer reviewed)
 
 Internet-expressed sentiment tries to capture the tone of the discussion in social media posts which can be written by the bank itself, public or private institutions or natural persons. \cite{fernandez2021} derive a sentiment indicator for the Mexican financial sector as a whole based on tweets, which captures sources of financial stress which are not reflected in quantitative risk measures and show that this indicator correlates with measures for financial risk. \cite{illia2021} analyse the relationship between the sentiment of tweets about a single bank and its daily business performance and find that tweet sentiment might affect bank performance when embedded in a larger conversation. The recent banking crisis in 2023 opened the discussion whether social media is a new factor which accelerates bank run behavior. Indeed first results in \cite{cookson2023} analysing the bank run on Silicon Valley Bank in 2023 indicate that social media sentiment could amplify bank run risk.
% \cite{mendoza2024}
% https://ieeexplore.ieee.org/stamp/stamp.jsp?arnumber=10648688
% https://www.sciencedirect.com/science/article/pii/S0378426620302314

\section{Measures for Financial Distress of Banks}



Since the riskiness of a bank cannot be measured, the empirical literature about financial distress in banks relies on indirect measures. Hence, different measures have been developed in the literature which should proxy the health of a bank. Some measures rely on accounting data of the corresponding banks. For example, \cite{sinkey1978} addresses the question of identifying problematic banks and observes that nearly all failed banks exhibited low net capital ratios. However, most banks with low net capital ratios did not fail. Other studies such as \cite{chiaramonte2016} analyse the predictive performance of the Altman Z-score in the banking sector. The main limitation of such measures is the relative low frequency due to the dependency on accounting data. Market measures in contrast are available in higher frequency and hence enable nowcasting the risk of banks. Market measures are derived from financial markets such as the stock or options market and try to get an implied proxy of the perceived risk of market participants. \cite{sarin2016} give an overview of market measures such as stock price volatility, option implied volatility, credit default swaps, the price to earnings ratio and preferred stock yields. They state that all indicators are only an imperfect proxy but looking at multiple different indicators enables the assessment of market beliefs.

%\begin{itemize}
%	\item \cite{coffinet2013}
%	\item \cite{avino2019}
%\end{itemize}
%
%Das muss unbedingt noch irgenwo rein (aus \cite{sarin2016}):
%It is noteworthy that, as Jeremy Bulow and Paul Klemperer (2013, 2015) and Andrew Haldane (2014) point out, measures of regulatory capital have historically not had much predictive power for bank failures.
% finde das irgendwie nicht so deutlich in diesen quellen... evtl besser doch weglassen


\section{Contribution to the Literature}

This study builds on existing studies about explaining and predicting risk proxies of banks. In addition to several simpler independent models, the model for analysing the determinants of CDS spreads from \cite{annaert2013} is extended by a news sentiment indicator, the model to predict CDS spreads for sovereign debt by \cite{cathcart2020} is modified to be suitable for banks and the research question of \cite{roeder2020} is analysed. The sentiment indicator is constructed for Swiss banks for an observation period from January 2022 to June 2023, hence including the Credit Suisse collapse, as well as for European G-SIBs for an observation period from September 2023 to October 2024. The study is designed in the perspective of financial supervision.

\cleardoublepage
