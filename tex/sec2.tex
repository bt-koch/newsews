\chapter{Sentiment Analytics for Supervision}\label{sec2}
\thispagestyle{empty}

First published in 1997 by the Basel Committee on Banking Supervision (BCBS), the 29 Core Principles for effective banking supervision, or the Basel Core Principles (BCPs), are the minimum global standard for the sound prudential regulation and supervision of banks and banking systems \citep{bis2024}. These BCPs can be categorised to BCPs 1-13 which define the powers, responsibilities and functions of supervisors and BCPs 14-29 about prudential regulations and requirements for banks. Relevant for this study are BCPs 8-13, which outline various tools and techniques aimed at facilitating robust on-site and off-site supervision of banks and banking groups. Particularly BCP 8 which defines the supervisory approach states, that the supervisor develops and maintains a forward-looking assessment of the risk profile of individual banks. \\

% mabye cite this (also) for BCPs: https://www.bis.org/bcbs/publ/d383.pdf

The turmoils in the banking sector in early 2023 highlighted the importance of this forward-looking risk assessment. As \cite{prenio2024} states, a lack of adequate resources and effective tools affects the ability of supervisory authorities for timely interventions. Besides investments in human resources, supervisory authorities are investing in innovative tools using novel techonology \citep{prenio2024}. \cite{broeders2018} defined the use of innovative technology by supervision agencies to support supervision as supervision technology, or suptech. According to \cite{prenio2024}, there is much hope that suptech helps to enhance supervisory ability, whereby the developments in generative artificial intelligence might provide the potential of suptech to be a transformative force in financial supervision. \\

\textcolor{blue}{
Facing this need for effective tools in banking supervision and opportunities offered by artificial intelligence, we introduce an indicator which measures the sentiment of the news media on banks under supervision. Given a corpus of news articles, we want to determine a single score which quantifies the sentiment of media coverage for the corresponding bank for a given time period. This score should range from -1 for the most negative to 1 for the most positive, with 0 indicating neutrality. To construct this indicator, the following problems must be addressed. First, the corpus must be filtered for articles which contain relevant information about the bank in question. Then, within these articles, the specific passages of the articles containing relevant content must be identified. All these relevant passages must then be classified as positive, neutral or negative and be assigned with a value accordingly. These classifications must then be combined to generate a single sentiment score of the corresponding news article and subsequently be further aggregated to generate a single sentiment score of the corresponding bank for a given time period. \\
}

\textcolor{blue}{
The resulting news sentiment indicator reduces the vast amount of published information to a single score and prevents the necessity of manually processing the news about multiple banks. Forward-looking risk assessment could benefit from this news sentiment indicator by proxying new information which is not or not yet reflected in other metrics. Additionally, besides being a proxy for a banks risk, news articles can also influence the behavior of market participants and hence the news sentiment indicator could be a leading indicator for moves in the financial markets (das muss ich noch umschreiben um mein Punkt besser r�ber zu bringen).
}

\cleardoublepage
