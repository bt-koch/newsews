\chapter{News Sentiment Indicator}\label{sec3}
\thispagestyle{empty}

\section{Overview}

As shown in previous literature, the sentiment in news articles can be a leading indicator for reactions in the stock market. Goal of this analysis is to construct the sentiment indicator from news media articles using a large language model (LLM) respectively a Bidirectional Encoder Representation from Transaction (BERT) model to estimate the sentiment from a given text. It is expected that a BERT model performs better than traditional methods such as bag-of-word algorithms in correctly classifying the sentiment of text due to its ability to understand context in text. Using news articles from different sources a sentiment indicator in daily and weekly frequency will be constructed. The goal of this indicator will be to deliver early warning for increased risk of a specific bank.


\section{Step 1: Preprocessing of News Articles}

For the articles in our sample, depending on the underlying data source, it is not guaranteed that an article contains relevant information in the context of our research question. Since the articles are queried from the data source by the banks name, its ticker or a common abbreviation, there will be articles in the sample which contain no information content about the financial stability of a bank but for example describe a sports event, for which the corresponding bank is mentioned as a sponsor. Additionally, it is possible that an article covers different topics or different companies and hence contains information which has no connection to the corresponding bank.\\

Hence, to avoid introducing noise through irrelevant articles, first a rule-based model is applied to identify relevant articles. First, the article is split into its paragraphs. Using pattern matching and predefined identifying patterns for the corresponding bank, we check for each paragraph if the bank is mentioned in it and remove paragraphs with no occurrence of the banks identifying pattern. Afterwards, again using pattern matching now with a predefined list of economic and financial keywords, we check whether the content of the article is relevant for our analysis by checking for the occurrence of a keyword in the paragraphs. Before we construct the sentiment indicator as described in step 2, we only consider articles which were classified as relevant as described before and feed only paragraphs in the construction workflow in which again the bank identifying pattern occurs.

\section{Step 2: Estimate Sentiment of Text} \label{sentim-classification}

Once Step 1 is finished, we have a list of paragraphs extracted from a specific article which we assume contain relevant information about the corresponding bank $b$ at time $t$. Next, for each paragraph, we want to assess whether the sentiment is positive, neutral or negative. For this classification, we use Bidirectional Encoder Representation from Transaction (BERT) models based on \cite{devlin2019}. \cite{devlin2019} use a masked language model (MLM) and next sentence prediction (NSP) as two unsupervised tasks to pre-train the BERT model. The MLM randomly masks 15\% of tokens in a sentence and predicts the  masked tokens using the remaining tokens in the sentence to obtain a bidirectional pre-trained model, respectively to understand the relationship between word in a sentence. To further understand sentence relationships, \cite{devlin2019} pre-train the BERT model with a next sentence prediction task. Hence, the main advantage of using a BERT model in this analysis compared to traditional methods such as lexicon-based models is the ability of the BERT model to understand context. Additionally, fine-tuning a pre-trained BERT model to perform a specific task does not require substantial task-specific modifications of the architecture and is hence relatively easy \citep{devlin2019}. To classify both german and english text, this analysis utilizes a BERT model which was fine-tuned for financial sentiment classification on the corresponding language.\footnote{Both models are available on \url{https://huggingface.co}. The english model is described in \cite{araci2019} and available on \url{https://huggingface.co/ProsusAI/finbert}. The german model is described in \cite{scherrmann2023} and available on \url{https://huggingface.co/scherrmann/GermanFinBert_SC_Sentiment}} Table \ref{tab:bert-examples} demonstrates the ability of the german BERT model to correctly classify the sentiment of the text given the context.

\begin{table}[H]
\caption{Examples of Sentiment Classification}
\label{tab:bert-examples}
\resizebox{\textwidth}{!}{%
\begin{tabular}{|l|l|}
\hline
input & classification \\ \hline
\textit{\begin{tabular}[c]{@{}l@{}}Diese Neuigkeiten �ber die schlechte Entwicklung der allgemeinen\\ Wirtschaftslage hat einen betr�chtlichen Einfluss auf die Credit Suisse.\end{tabular}}  & negative       \\ \hline
\textit{\begin{tabular}[c]{@{}l@{}}Diese Neuigkeiten �ber die schlechte Entwicklung der allgemeinen\\ Wirtschaftslage hat keinen betr�chtlichen Einfluss auf die Credit Suisse.\end{tabular}} & neutral        \\ \hline
\textit{\begin{tabular}[c]{@{}l@{}}Diese Neuigkeiten �ber die schlechte Entwicklung der allgemeinen Wirtschafts-\\lage hat einen betr�chtlich vorteilhaften Einfluss auf die Credit Suisse.\end{tabular}} & positive       \\ \hline
\end{tabular}%
}
\end{table}

The sentiment $s$ towards bank $b$ at time $t$ of each paragraph $p$ in the article $a$ in this list is classified by the pre-trained BERT model as follows:

\begin{equation}
s_{b,t,a,p} = \begin{cases}
	-1	& \text{if text is mainly classified as negative}\\
	0 	& \text{if text is mainly classified as neutral} \\
	1	& \text{if text is mainly classified as positive}
\end{cases}
\end{equation}


\section{Step 3: Construction of Sentiment Indicator}

After the sentiment of all paragraphs $p \in P$ in article $a$ about bank $b$ at time $t$ is classified as described in section \ref{sentim-classification}, the overall sentiment of article $a$ ($s_{b,t,a}$) is calculated as the arithmetic mean of all corresponding paragraphs $p$. To define the overall sentiment towards bank $b$ at time $t$, the sentiment is further aggregated by the arithmetic mean of the sentiment $s_{b,t,a}$ for all articles $a \in A$ about bank $b$ at time $t$:

 \begin{equation} \label{eq:sentind}
 	s_{b,t} = \frac{1}{A} \sum^A_{a=1} s_{b,t,a} \quad\text{whereby}\quad s_{b,t,a} = \frac{1}{P} \sum^P_{p=1} s_{b,t,a,p}
 \end{equation}
 
 As the literature shows, sentiment scores tend to be noisy, which is why smoothing techniques or aggregation on higher frequency are applied to reduce the noise observed. In this analysis, when working in weekly frequency, we assume that the noise is adequately reduced. We assume that information contained in articles before $t$ remain relevant at time $t$, whereby the relevancy is decreasing over time. When working in weekly frequency, we incorporate this assumption into our models by including lagged values of the sentiment indicator $s_{b,t}$ into our models. However, when working in daily frequency, we construct the sentiment indicator $s\_wma_{b,t}$ as a $N$-period backward-looking weighted moving average:
 
\begin{equation} \label{eq:sentind_wma}
	s\_wma_{b,t} = \frac{1}{\sum^N_{n=1}n} * \sum^{N-1}_{n=0}(N-n)*s_{b,t-n}
\end{equation}



\cleardoublepage
