\chapter{Experimental Analysis}\label{sec5}
\thispagestyle{empty}

Hier noch gem�ss PQM Richtlinien: Ziel, Motivation (das habe ich ja eig. schon) und Aufbau des Kapitels (ein Satz je Abschnitt). muss daf�r ein overview section gemacht werden? hier auch noch auf git repo hinweisen

\section{Data}

For this analysis, news media articles were retrieved from either the Swissdox@LiRI database or from the Eikon Data application programming interface (API). Depending on the queried bank, one of those providers were used. \\

Swissdox@LiRI is a database which includes about 23 million media articles from Swiss media sources, mainly from the German and French speaking parts of Switzerland. It provides an interface specifically designed for big data applications and hence allows to query the whole database with the option to filter for language, time interval, keywords, sources and document types and returns all matching news articles in a machine-readable format. The whole content of the article is available. For this analysis, a query for each publicly traded Swiss bank was submitted to the database, whereby the database was filtered using the banks name, stock ticker or a common abbreviation of the bank name as keywords for a time interval starting in January 2022 until June 2023. Only German-language articles were used for the analysis however. \\

The sample of publicly traded Swiss banks was extended to all European Globally Systematically Important Banks (G-SIBs) from the 2023 list, which were identified by the Financial Stability Board in consultation with the Basel Committe on Banking Supervision and national authorities. To retrieve news articles about the European G-SIBs, the Eikon Data API as a second provider for news media articles was used, which provided the option to request news articles from their news feed. This provider also offers the whole content of news articles in a machine-readable format. However, since the interface is not specifically designed for big data applications, the procurement of each item requires a separate API request. Therefore, corresponding API limits are quickly reached, which is why only english articles which were labeled as significant news by an algorithm by Eikon were requested. Furthermore, if there were more than 5 articles correspondent to a specific bank on a given day, only five articles were queried which were chosen randomly. HIER SCHREIBEN WIE H�UFIG DAS MIT MEH ALS F�F DER FALL WAR. Furthermore, using the Eikon Data API, only news articles from the past 15 months can be retrieved. Thus, the observation period start in September 2023 and spans until October 2024.

\section{Experimental Design and Hypotheses} \label{sec:hypotheses}

First, the analysis focuses on Credit Default Swap (CDS) spreads as a proxy of the risk associated with a bank. We start by analyzing whether the sentiment indicator is a determinant of the changes in CDS spreads. This analysis is based on the work of \cite{annaert2013}, which will be extended by the inclusion of the sentiment indicator as a potential determinant. \cite{annaert2013} try to explain and not to predict changes in CDS spreads using contemporaneous explanatory variables. The explanatory variables include the risk free rate, leverage and the equity volatility as credit risk variables, the bid-ask spread as liquidity variable and the term structure slope, swap spread, corporate bond spread, market return and market volatility as business cycle and market wide variables. The definition of these explanatory variables can be found in the Online Appendix TODO!!!. \cite{annaert2013} run their analysis for 32 listed banks from the euro area in the period from 2004 to 2010. HIER NOCH MEHR �BER IHRE FINDINGS SCHREIBEN. VORALLEM DASS DIE GESCH�TZTEN KOEFFS �BER DIE ZEIT NICHT STABIL SIND. \\		

Now, we introduce the sentiment indicator as described in equation \ref{eq:sentind} as a new explanatory variable in the model of \cite{annaert2013}. Due to data availability the explanatory variable bid-ask spread from \cite{annaert2013} is excluded from our analysis. The multivariate panel regression model is then given as follows:

\begin{equation}
	cdsspread_{b,t} = \alpha_b + \sum_{k=1}^K\beta_{k}x_{b,k,t} + \theta s_{b,t} + \sum_{g=1}^G\gamma_gz_{g,t}+e_{b,t}
\end{equation}

whereby $b$ identifies the bank and $t$ the time period. Note that we use the time-varying bank-specific explanatory variables ($x_{b,k,t}$) and time-varying common explanatory variables ($z_{g,t}$) but extend the model by the time-varying bank-specific sentiment indicator $sentiment_{bt}$. Using this regression model, we test following hypothesis:

\begin{hyp}
	The sentiment indicator is a determinant of CDS spreads.
\end{hyp}

Hence, we expect and test whether the coefficient for sentiment $\theta$ is different from zero. More precisely, we expect $\theta<0$, meaning that negative sentiment $s_{b,t}<0$ is associated d with increased CDS spreads and hence higher risk of the corresponding bank $b$. Since CDS are traded solely on UBS and Credit Suisse in the sample of Swiss banks, we extend our sample to include all European G-SIBs. Due to the availability of data, the observation period spans from September 2023 to October 2024. The corresponding sentiment indicator is based on news obtained from the Eikon Data API. Additionally, the model is also evaluated for the sample Credit Suisse and UBS using the sentiment indicator derived from Swissdox@LiRI for the observation period from January 2022 until June 2023. \\

The main interest of this analysis is however, whether the indicator constructed from news articles could serve as an early warning signal for distressed banks. Hence, rather than evaluating whether the sentiment indicator determines the current CDS spread, we are interested whether the indicator can predict future changes in risk proxies such as the CDS spread, the maximum drawdown or the equity volatility of the banks in our samples. \\

First, we assess the predictive performance of the news sentiment indicator on CDS spread. We follow \cite{cathcart2020}, which find that their news sentiment indicator has predictive power for CDS spreads on sovereign debt. They construct a global news sentiment indicator using the Thomson Reuters News Analytics (TRNA) database and focus on 25 developed and less developed countries from various geographies from 2003 to 2014. Besides finding that their sentiment indicator is a determinant of CDS spreads using panel regression techniques, they \cite{cathcart2020} find that their news sentiment indicator can predict CDS spreads using panel vector autoregression methods. \\

Based on this results, we adjust the panel vector autoregressive model proposed by \cite{cathcart2020} to be suitable for estimating the relationship between the news sentiment of the banks in our sample on their corresponding CDS spreads. Note that our analysis has a bank specific independent variable $s_{b,t}$ for the news sentiment. Additionally, since high frequency bank-specific controls based on fundamentals or other data sources despite market data are difficult to include, we have no further bank specific controls in the regression model. HERE WHAT ABOUT FIXED EFFECTS? Hence, the model is given as follows:

\begin{equation} \label{eq:panelvar-cds-exogsenti}
	cdsspread_{b,t} = \alpha_b + \sum_{k=1}^K\beta_{k}x_{k,t-1} + \sum^5_{\tau=1} \delta_{\tau}cdsspread_{b,t-\tau} + \sum^5_{\tau=1}\theta_{\tau}s_{b,t-\tau} + e_{b,t}
\end{equation}

whereby $x_{k,t-1}$ are the explanatory variables stock market, volatility premium, term premium, treasury market, investment grade and high yield following \cite{cathcart2020}. The other explanatory variables from \cite{cathcart2020} are not included since they are only suitable when analysing sovereign debt and are not replaced again since bank specific controls in high frequency are difficult to obtain. FIXED EFFECTS? Following \cite{cathcart2020}, we include the CDS spread and the sentiment indicator lagged up to five periods in the model. Using this specification of the vector autoregressive model we test following hypothesis:

\begin{hyp}
	The sentiment indicator predicts CDS returns.
\end{hyp}

Hence, we expect and test whether the coefficient $\theta_1$ is non-zero or more specifically $\theta_1<0$ for the model with one lag ($T=1$), meaning that negative news sentiment increase CDS spreads in the next period and hence indicate increased perceived risk of the corresponding bank $b$ in the next period. For the model specification with five lags ($T=5$), we expect that the coefficients in total are non-zero and indicate that negative sentiment increases perceived risk. However, due to the in the literature already discussed over- and underreaction to news, we might expect different magnitudes and sign changes in the coefficient for sentiment, which would be explained by a overreaction to new news. \\

In the model as described in equation \ref{eq:panelvar-cds-exogsenti}, the sentiment indicator is treated as an exogenous variable. However, there is high potential for reverse causality. News content might affect the behavior of market participants which might change CDS spreads. In the same way, changes in CDS spreads might affect the news reporting about the corresponding bank. Because of this potential bidirectional causality we also fit the model as described in equation \ref{eq:panelvar-cds-exogsenti} with the sentiment indicator as a second endogenous variable while the remaining controls remain exogenous. This further allows to analyze the relationship between the CDS spreads and the sentiment indicator as an impulse response function (IRF). We therefore further analyze following hypotheses:

\begin{hyp}
	There is a bidirectional relationship between CDS spreads and the news sentiment indicator.
\end{hyp}

\begin{hyp}
	The sentiment indicator supports the theory of over- and underreaction.
\end{hyp}

We test if there is a bidirectional hypothesis by conducting a Granger Causality test as well as looking at the IRF. To test the over- and underreaction hypothesis, we look at the effect of news sentiment on CDS spreads using the corresponding IRF. \\

The main problem of the previous regression models is, that it requires CDS spreads on the banks to be analyzed. However, there are no traded CDS on the majority of Swiss banks. Hence, to include the remaining banks of our Swiss sample in the analysis, we need to use another proxy for the risk perceived in the financial markets about the corresponding banks. Since all banks are publicly traded, risk proxies constructed from the corresponding stock price allows to include all banks to be included in the analysis without high demands on data availability. Hence, the remaining models focus on risk proxies derived from the stock price and are fitted on the whole sample of Swiss banks and European G-SIBs. \\

First, we assume that the sentiment of the news media is reflected in the stock price of the corresponding bank in the foreseeable future. As the literature has shown, the effect of news sentiment is most pronounced for media pessimism which is able to predict falling stock prices (for example as in \cite{tetlock2007}). Additionally, from the perspective of banking supervision, we care most about increased risk. Hence, we use the maximum drawdown over the next 14 days of the corresponding bank's stock price as a risk proxy. For this analysis, we analyse the data in daily frequency and hence use the weighted moving average of the news sentiment as described in equation \ref{eq:sentind_wma}. We therefore fit following panel autoregressive model:

\begin{equation} \label{eq:mdd}
	mdd_{b,t} = \alpha_b + \sum_{\tau=1}^5\delta_\tau mdd_{b,t-\tau} + \theta s\_wma_{b,t-1} + e_{b,t}
\end{equation}

whereby $mdd_{b,t}$ is the $t+14$ maximum drawdown of the stock price of bank $b$ at time $t$. Again, as in the model as described in equation \ref{eq:panelvar-cds-exogsenti}, the maximum drawdown as the independent variable is included as lagged control variables for up to five lags in the model. HERE SAY SOMETHING ABOUT MISSING CONTROL VARIABLES. HERE REPORT REASON AND METHOD FOR ADJUSTED SMA. Using this model, we check following hypothesis:

\begin{hyp}
	The sentiment indicator predicts the maximum drawdown of a banks stock price.
\end{hyp}

Again, we expect and test whether the coefficient to be non-zero or more specifically $\theta<0$, indicating that negative sentiment increases the maximum drawdown. \\

Another risk proxy which we can directly derive from the stock price is the corresponding volatility. We assume that negative news sentiment increases risk and uncertainty about the corresponding bank in following periods. Hence, we expect that negative news sentiment is associated with higher volatility in the following periods. To analyze this relationship, we fit both a generalised autoregressive conditional heteroscedasticity (GARCH) model with covariates, GARCH-X, as well as a heterogeneous autoregressive (HAR) model with covariates. Following the implementation of the GARCH-X model as in \cite{sucarrat2021} and the HAR model as in \cite{boudt2022}, only one additional covariate can be included in the model. Hence, we have following representation of the GARCH-X model: 

\begin{equation}
	\sigma^2_{b,t} = \alpha_b + \beta_{b,1}e^2_{t-1} + \beta_{b,2}\sigma^2_{t-1} + \theta_bs\_wma_{b,t-1}
\end{equation}

whereby $\sigma^2_{b,t}$ is the realised volatility, $e^2_{t-1}$ is the squared shock in $t-1$ and $s\_wma$ is the weighted moving average of the news sentiment indicator as described in equation \ref{eq:sentind_wma}. Additionally, we have following representation of the HAR model as well:

\begin{equation}
	\sigma^2_{b,t} = \alpha_b + \beta_{b,1}\sigma^2_{b,t-2:t} + \beta_{b,2}\sigma^2_{b,t-6:t} + \beta_{b,3}\sigma^2_{b,t-21:t} + \theta_b s_{b,t-1} + e_{b,t}
\end{equation}

whereby $\sigma^2_{b,t}$ is the realised volatility at $t$. We include the volatility of the stock price of the past 3, 7 and 22 days ($\sigma^2_{b,t-2:t}$, $\sigma^2_{b,t-6:t}$ and $\sigma^2_{b,t-21:t}$) as well as the weighted moving average of the sentiment indicator $s\_wma_{b,t-1}$ according to equation \ref{eq:sentind_wma}.

\begin{hyp}
	The sentiment indicator predicts stock price volatility.
\end{hyp} 

We expect and test whether the coefficient for the sentiment indicator is different from zero, more specifically we expect $\theta_b>0$ which would indicate that negative sentiment increases stock price volatility. Note that both the GARCH-X model and the HAR model can only be fitted for one bank at once. Hence, we do not estimate the whole panel, rather we focus on Credit Suisse and UBS as the two biggest banks in the Swiss sample, as we expect highest media coverage and market activity for these banks.

\section{Evaluation of Hypotheses} \label{sec:results}

% hypothesis 1:
First, according to hypothesis 1 we check whether the sentiment indicator is a determinant of CDS spreads. In table \ref{tab:cdsdet} we show the results of the panel regression of CDS spreads following \cite{annaert2013}, whereby the main interest is in the estimated coefficient for the sentiment. As we can see, the estimated coefficient is not statistically different from zero. Still, as expected, the coefficient is negative which would mean that negative sentiment is associated with increased risk of the corresponding bank. Noticeable is the difference in magnitude of the coefficient and its standard error between the two samples. Given the results of the panel regression following \cite{annaert2013}, we cannot confirm hypothesis 1 that the current news sentiment of a bank is a determinant of the corresponding CDS spread. \\

\input{tables/cdsdet.tex}

% hypothesis 2,3 and 4:
Next, by evaluating the output of the panel vector autoregressive model of CDS spreads following \cite{cathcart2020}, we check whether the sentiment indicator has predictive power on CDS spreads. Table \ref{tab:cdspvar} shows the model according to equation \ref{eq:panelvar-cds-exogsenti} fitted on the European banks in Model 1 and on the Swiss banks in Model 2. For the European sample, we do not observe any coefficients of the different lagged values of the sentiment indicator which are statistically different from zero. For the Swiss sample, the coefficient for the sentiment indicator of lag order one is statistically different from zero and of economically significant magnitude. The sentiment indicators of higher lag order are still not statistically significantly different from zero. Still, the sign and magnitude might indicate a possible correction of the strong reaction of the lag order of zero. Hence, we can only partially accept hypothesis 2 that the sentiment indicator has predictive power for CDS spreads.

\input{tables/cdspvar.tex}

To further analyse this hypothesis, we conduct a Granger causality test to assess whether the sentiment indicator helps to predict the CDS spreads and vice versa. Note that this test is conducted for one bank at a time without additional control variables. We run the Granger causality test for both Credit Suisse and UBS and report the results in table \ref{tab:granger_weekly}. The results of this test suggests that the news sentiment indicator Granger-causes the CDS spreads for Credit Suisse, for UBS only the specification with 2 lags suggests this Granger-causality. Additionally, there seems to be no Granger-causality of CDS spreads on the news sentiment. Hence, this results do not support hypothesis 3 of the bidirectional relationship between CDS spreads and the news sentiment indicator but rather support that there is Granger-causality of news sentiment on CDS spreads only.

\input{tables/granger_weekly.tex}

Figure \ref{fig:oirfs} further support this conclusion. As we can see, the orthogonalized impulse response functions (IRFs) suggest that shocks in the news sentiment are followed by a reaction in the CDS spreads while a shock in the CDS spreads do not seem to be followed by a reaction in the news sentiment. Additionally, in the orthogonalized IRFs show the over- and underreaction which is consistent with theoretical literature (as ...) as well as empirical studies (as in \cite{cathcart2020}). Hence, this results support the hypothesis 4 that the sentiment indicator supports the theory of over- and underraction in the financial markets. \\

% to do: make plot by myself and wrap this tex code in R function
\begin{figure}[h!]
    \centering
    \includegraphics[width=0.75\textwidth]{images/oirfs.png}
    \caption{Orthogonalized impulse response function}
    \label{fig:oirfs}
\end{figure}


% hypothesis 5:
Having analysed the relationship between the sentiment indicator as proposed in equation \ref{eq:sentind} and CDS spreads, we now turn to the hypothesis whether the sentiment indicator as proposed in \ref{eq:sentind_wma} has predictive performance for the maximum drawdown over the next 14 days of the corresponding bank. For the European G-SIBs, the estimated coefficients for the sentiment indicator are not statistically different from zero, hence the results of the model according to equation \ref{eq:mdd} are not reported. Table \ref{tab:mddpvar_swiss} reports the results of this regression for the Swiss sample. While the coefficient of the unadjusted sentiment indicator is not statistically significantly different from zero, the coefficient of the adjusted sentiment indicator is. The results indicate that negative sentiment is associated with a higher drop in the stock price represented as a lower value of the maximum drawdown according to equation TODO. Hence, the results of the Swiss Bank sample support the hypothesis 5 that the news sentiment indicator has predictive power of the maximum drawdown. \\

\input{tables/mddpvar_swiss.tex}

% hypothesis 6:
Lastly, when evaluating both models for predicting the volatility of the banks stock price, neither the GARCH-X model as reported in table \ref{tab:garchx} nor the HAR model as reported in table \ref{tab:har} resulted in estimated coefficients for the sentiment indicator which are statistically different from zero. Hence, this analysis does not support the hypothesis 6 that the sentiment indicator has predictive performance in forecasting future stock price volatility of corresponding banks.

\input{tables/garchx.tex}

\input{tables/har.tex}



%\section{Evaluation of Predictive Performance, out of sample analysis}
%
%here i guess i will just focus on champion model \\
%
%then also look at credit suisse at crisis \\
%
%visualize pred vs actual for cs and ubs \\


\section{Discussion}

The results in section \ref{sec:results} only partly support our hypotheses stated in section \ref{sec:hypotheses}. Particularly striking is the difference between the results of the European G-SIB sample and the Swiss Banks sample. One difference between the samples is the observation period used for the analysis. As the results of \cite{annaert2013} have shown, estimated coefficients and their significants are not stable over time. In particular, their results suggest that the variables included in their analysis were more influential in time of crisis. The Swiss Bank sample does include the 2023 banking crisis while the European sample does not, which might be a reason for the difference in results. \\

Another big difference is the news article database the corresponding news sentiment articles are constructed upon. The news sentiment indicator constructed for the Swiss banks sample is built on much more articles which, through aggregation, might reduce the noise by giving less weight to potential articles which do not contain much impactful information about the state of the corresponding bank. Since the general public which consumes mainstream media as contained in the database used for the Swiss sample does not care as much about recent developments as the reader of news provided by specialised vendors as Refinitv, only highly relevant information might be published in the mainstream media. Hence, news articles published from mainstream media as in the Swiss banks sample might contain information of greater importance compared to financial news in average. \\

The results in this analysis can be interpreted as explorative studies for finding whether the content of financial news articles can be quantified and used as a predictive indicator in applications relevant for banking supervision or risk management. We can see that for some periods, samples and risk proxies there might be predictive performance in the content of media articles. To build a reliable indicator, more detailed analysis regarding sample size, observation period and robustness is required. As we have seen in the analysis of hypothesis 5, there might also be the need of further adjusting the indicator by adapting the classification rule to distinguish more clearly between positive, neutral and negative signals. \\

Finally, to build an early warning indicator, we would need to implement a decision rule on when the corresponding bank should be classified as risky enough to be further monitored by the supervisor. If adverse events of regular severity are occurring and the financial markets are reacting accordingly, there should not be a signal for the supervisor. For example, if quarterly results are worse than expected, which could be reflected in the news coverage as well as the risk proxies, a financially sound bank should not become financially distressed. Thus, a threshold of the corresponding risk proxy could be defined and if the predicted value crosses this threshold could be defined as the decision rule to initiate a more in-depth monitoring process. There might also be a threshold which could be implemented which decides whether a predicted increase in the corresponding risk proxy could be a normal market reaction or a sign of increased financial distress. \\

The sentiment indicator itself could also be constructed in a different way such that the indicator itself could be used for a decision rule. The sentiment indicator as of now does not distinguish on news on higher or lower relevancy for a supervisory context. This could be done by training a LLM by providing a metric for each textual input which indicates the level of relevancy. Another possibility would be to define classes of topics such as legal news which are highly relevant in the supervisory context and construct the sentiment indicator only on news about these predefined categories. Additionally, the construction of the sentiment indicator could further be extended such that the textual inputs could be classified not only as negative but also in slightly negative, negative or highly negative (for example by assigning values -1, -2 and -3 accordingly). For this, it would be required to train a novel LLM which could classify the relevancy and severity of negative news in a supervisory perspective. Given the methods used in this analysis, one could construct the sentiment indicator by multiplying the assigned value of -1, and 1 for positive news with the metric provided by the LLM about how sure it is about its classification (HERE WRITE THIS LESS LIKE ITS A HUMAN OR SOMETHING.) \\

Furthermore, the analysis could be extended by predicting spreads between the risk proxies between the different banks. For example, one could define the bank with the risk proxy which indicates the least risk or the average risk proxy as a benchmark and analyse the developments of the spreads from the banks to this benchmark and if they can be predicted using a sentiment indicator which itself could be the spread between the sentiment towards the specific bank compared to all banks in general. This might reveal if the risk of a specific bank deteriorates abnormally in comparison to its peers. \\

Talk here also about PCA

\cleardoublepage
