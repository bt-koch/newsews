\chapter{Experimental Analysis}\label{sec5}
\thispagestyle{empty}

\section{Data}

For this analysis, news media articles were retrieved from either the Swissdox@LiRI database or from the Eikon Data application programming interface (API). Depending on the queried bank, one of those providers were used. \\

Swissdox@LiRI is a database which includes about 23 million media articles from Swiss media sources, mainly from the German and French speaking parts of Switzerland. It provides an interface specifically designed for big data applications and hence allows to query the whole database with the option to filter for language, time interval, keywords, sources and document types and returns all matching news articles in a machine-readable format. The whole content of the article is available. For this analysis, a query for each publicly traded Swiss bank was submitted to the database, whereby the database was filtered using the banks name, stock ticker or a common abbreviation of the bank name as keywords for a time interval starting in January 2022 until June 2023. Only German-language articles were used for the analysis however. \\

The sample of publicly traded Swiss banks was extended to all European Globally Systematically Important Banks (G-SIBs) from the 2023 list, which were identified by the Financial Stability Board in consultation with the Basel Committe on Banking Supervision and national authorities. To retrieve news articles about the European G-SIBs, the Eikon Data API as a second provider for news media articles was used, which provided the option to request news articles from their news feed. This provider also offers the whole content of news articles in a machine-readable format. However, since the interface is not specifically designed for big data applications, the procurement of each item requires a separate API request. Therefore, corresponding API limits are quickly reached, which is why only english articles which were labeled as significant news by an algorithm by Eikon were requested. Furthermore, if there were more than 5 articles correspondent to a specific bank on a given day, only five articles were queried which were chosen randomly. HIER SCHREIBEN WIE H�UFIG DAS MIT MEH ALS F�F DER FALL WAR.




\section{Experimental Design and Hypotheses}

%Describe and state hypothesis. Describe model setup. \\

First, we analyze whether the sentiment indicator is a determinant of changes in CDS spreads. This analysis is based on the work of \cite{annaert2013}, which will be extended by the inclusion of the sentiment indicator as a potential determinant. \cite{annaert2013} try to explain and not to predict changes in CDS spreads, for which they use contemporaneous explanatory variables which can be categorized as credit risk variables, marketability indicators and market wide factors. When analyzing weekly data of 32 listed euro area banks in the period 2004-2010, they DESCRIBE THEIR FINDING HERE IN TERMS OF R SQUARED ETC. Now, we introduce the sentiment indicator as described in equation \ref{eq:sentind} as a new explanatory variable in the model of \cite{annaert2013}. Due to data availability the explanatory variable \textit{bid ask spread} from \cite{annaert2013} is excluded from our analysis. The regression model is then given as follows:

\begin{equation}
	y_{it} = \alpha_b + \sum_{k=1}^K\beta_{bk}x_{bkt} + \beta_{K+1}sentiment_{bt} + \sum_{g=1}^G\gamma_gz_{gt}+e_{jt}
\end{equation}

whereby $b$ identifies the bank and $t$ the time period. Note that we use the time-varying bank-specific explanatory variables ($x_{bkt}$) and time-varying common explanatory variables ($z_{gt}$) but extend the model by the time-varying bank-specific sentiment indicator $sentiment_{bt}$. Using this regression model, we test following hypothesis:

\begin{hyp}
	The sentiment indicator is a determinant of CDS spreads.
\end{hyp}

We will run the regression with and without the sentiment indicator and assess the significance and the change in the adjusted $R^2$ to test this hypothesis. Since CDS are only traded on UBS and Credit Suisse, we extend our sample to include all European banks which are classified as global systemically important (G-SIB). Additionally, the source for the corresponding news articles are only available for the past 15 months, which is why the observation period is limited from September 2023 to October 2024.\\

However, to use the sentiment indicator as an early warning indicator, we need to assess its predictive power. We follow \cite{cathcart2020}, which find that news sentiment explains and predict CDS returns on countries. Additionally, \cite{cathcart2020} show that the sentiment indicator is a mixture of noise and information. We adjust the panel vector autoregression proposed by \cite{cathcart2020} to be suitable for estimating the model for banks instead of for sovereign credit risk:

\begin{equation}
	y_{it} = \alpha_i + \beta_{ik}X_{kt-1} + \sum^5_{l=1} \delta_{il}Y_{it-l} + \sum^5_{l=1}\gamma_{il}sentiment_{t-l} + e_{it}
\end{equation}

whereby HERE DESCRIBE THE VARIABLES IN THE REGRESSION MODEL. 

\begin{hyp}
	The sentiment indicator predicts CDS returns.
\end{hyp}

do this with significance and this f test for all lags. run following regressions:

\begin{itemize}
	\item pvar only sentiment
	\item pvar lagged sentiment
	\item pvar sentiment and stock returns
	\item pvar lagged sentiment and stock returns
\end{itemize}

%\begin{hyp}
%	The sentiment indicator supports the theory of over- and underreaction.
%\end{hyp}
%
%do this by analyzing IRFs. \\

Until now, the focus of this analysis was on CDS which are at the moment of this study only traded on UBS as the only Swiss bank. Hence, to assess whether the news sentiment indicator could be used in Swiss banking supervision, we need to use a risk proxy other than CDS. Since this study focuses on publicly traded banks, in the following it is analyzed whether the sentiment indicator has predictive power for the maximum drawdown and stock price volatility. 

Using the same explanatory variables, we assess whether the sentiment indicator is able to predict the maximum drawdown of the corresponding bank. BLABLA

\begin{equation}
	mdd model
\end{equation}

Again variable descr. Hence hypo:

\begin{hyp}
	The sentiment indicator predicts the maximum drawdown of a banks stock price.
\end{hyp}



 stock price volatility and maximum drawdown. Following \cite{audrino2020} .... write that not all explainable vars used.

HAR \cite{boudt2022}

\begin{equation}
	RV_{t+1} = \beta_0 + \beta_1RV_t + \beta_2RV_{t-4:t} + \beta_3RV_{t-21:t} + \beta_4sentiment_{t} + \epsilon_{t+1}
\end{equation}

whereby BLABLABLA. Hence hypo:

GARCHX: \cite{sucarrat2021}


\begin{equation}
	\sigma^2_t = \omega + \alpha_1\epsilon^2_{t-1} + \beta_1\sigma^2_{t-1} + \lambda_1x_{1,t-1}
\end{equation}

\begin{hyp}
	The sentiment indicator predicts stock price volatility.
\end{hyp} 



%\noindent
%First, some preliminary results of simle VAR of just CDS on sentiment: \\
%
%\noindent
%Using refinitiv sample:\\
%Significant and as expected positive coefficient for sentiment.
%
%\begin{figure}[h]
%\includegraphics[width=0.75\textwidth]{images/var_refinitiv_preliminary_results.png}
%\end{figure}
%
%\newpage
%\noindent
%Using swissdox sample:\\
%Significant but negative coefficient for sentiment?
%\begin{figure}[h]
%\includegraphics[width=0.75\textwidth]{images/var_swissdox_preliminary_results.png}
%\end{figure}
%
%\newpage
%\noindent
%Panel OLS as in \cite{cathcart2020}
%
%\begin{figure}[h]
%\includegraphics[width=0.75\textwidth]{images/cathcart_panel_ols.png}
%\end{figure}
%
%\newpage
%\noindent
%Panel VAR as in \cite{cathcart2020}\\
%Similar to panel OLS except coefficient for termpremium.
%
%\begin{figure}[h]
%\includegraphics[width=0.65\textwidth]{images/cathcart_panel_var.png}
%\end{figure}
%
%Next, use VAR for prediction
%
%\newpage

\section{Evaluation of Hypotheses}

%Show regression outputs here, impulse response functions for time series models, metrics for prediction performance, robustness checks etc. \\
%
% paragraph 1: hypothesis 1 ----
Table \ref{tab:cdsdet} shows the panel regression following \cite{annaert2013}, whereby the main interest is in the estimated coefficient of $sentiment$ in the regression on changes in CDS spreads. As we can see, the estimated coefficient is not statistically significantly different from zero. As expected, the coefficient is negative which means that negative sentiment increases risk associated with the bank. Noticeable is the difference in magnitude of the coefficient and its standard error.

\input{tables/cdsdet.tex}

Given the results of the panel regression following \cite{annaert2013} does not confirm our hypothesis that the current news sentiment of abank is a determinant of the corresponding CDS spread. \\

% paragraph 2: hypothesis 2 ----



%\input{tables/cdspvar.tex}

\input{tables/cdspvar_euro.tex}

\input{tables/cdspvar_swiss.tex}

hier dann noch ausf�hren dass nur 2 groups in sample. hier dann auch regressionen von gruppen separat zeigen und unterschiede feststellen inkl. m�glicher gr�nde. \\

hier w�re cool wenn ich IRFs zeigen k�nnte und eventuell auch die "behavioral story" sehen w�rde -> zuerst �berreaktion, dann korrektur auf h�herem niveau w�re sch�n zu sehen

% paragraph 3: hypothesis 3 ----

\input{tables/mddpvar.tex}

hier dann erw�hnen dass daily data hence wma verwendet wurde. erw�hnen dass auch signifikant ohne cs in sample

% paragraph 4: hypothesis 4 ----

\input{tables/garchx.tex}

\input{tables/har.tex}

% paragraph 5: hypothesis 5 ----

\section{Evaluation of Predictive Performance}

here i guess i will just focus on champion model \\

then also look at credit suisse at crisis \\

visualize pred vs actual for cs and ubs \\

\newpage

\section{Discussion}

hier auch noch hinweisen auf unterschied zwischen refinitiv vs swissdox. \\

klar ist sample period anderst. aber eventuell auch unterschiede zwischen den quellen? im sinne von enthalten dedicated financial news articles andere informationen als main stream media? im sinn von eventuell ist relevanz von information in mainstream h�her da "regul�re entwicklungen" grosse population ncith interessiert.

paragraph 1: result 1 \\

paragraph 2: result 2 \\

paragraph 3: result 3 \\

paragraph 4: result 4 \\

paragraph 5: result 5



\cleardoublepage
