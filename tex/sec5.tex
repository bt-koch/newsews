\chapter{Experimental Analysis}\label{sec5}
\thispagestyle{empty}

\section{Overview}

Turning now to the implementation of the sentiment indicator as discussed in \mbox{Chapter~\ref{sec4}}, in \mbox{Section~\ref{sec:data}} we describe the dataset used for constructing the news sentiment indicator as well as the sample of banks considered in our analysis. We present the hypotheses to be analysed in \mbox{Section~\ref{sec:hypotheses}}. In \mbox{Section~\ref{sec:results}} we evaluate the results of testing for these hypotheses. We conclude by discussing the benefits and limitations of this study in \mbox{Section~\ref{sec:discussion-sec5}}. The implementation and instructions for replication is available at \url{https://github.com/bt-koch/newsews/}. In accordance with the terms of use, the data however is not available for sharing.

\section{Data} \label{sec:data}

Depending on the queried bank, news media articles are retrieved from either the Swissdox@LiRI database or from the Eikon Data application programming interface (API). Swissdox@LiRI is a database which includes about 23 million media articles from Swiss media sources, mainly from the German and French speaking parts of Switzerland. It provides an interface specifically designed for big data applications and hence allows to query the whole database with the option to filter for language, time interval, keywords, sources and document types and returns all matching news articles in a machine-readable format. The whole content of the article is available. For this analysis, a query for each publicly traded Swiss bank is submitted to the database, whereby the database is filtered using the banks name, stock ticker or a common abbreviation of the bank name as keywords for a time interval starting in January 2022 until June 2023. Only German-language articles are used for the analysis however. \\

The sample of publicly traded Swiss banks is extended to all European Globally Systematically Important Banks (G-SIBs) from the 2023 list published by the Financial Stability Board (FSB), which are identified by the FSB in consultation with the Basel Committe on Banking Supervision as well as with national authorities. To retrieve news articles about the European G-SIBs, the Eikon Data API as a second provider for news media articles is used, which provides the option to request news articles from their news feed. This provider also offers the whole content of news articles in a machine-readable format. However, since the interface is not specifically designed for big data applications, the procurement of each item requires a separate API request. Therefore, corresponding API limits are quickly reached, which is why only english articles labeled as significant news by an algorithm by Eikon are requested. Furthermore, if there were more than five articles correspondent to a specific bank on a given day, only five articles are queried which were chosen randomly. This rule was necessary for 13.3\% of all requested days. Furthermore, using the Eikon Data API, only news articles from the past 15 months can be retrieved. Thus, the observation period starts in September 2023 and spans until October 2024.

\section{Experimental Design and Hypotheses} \label{sec:hypotheses}

First, the analysis focuses on Credit Default Swap (CDS) spreads as a proxy of the risk associated with a bank. We start by analysing whether the sentiment indicator is a determinant of the changes in CDS spreads. This analysis is based on the work of \cite{annaert2013}, which will be extended by the inclusion of the sentiment indicator as a potential determinant. \cite{annaert2013} try to explain and not to predict weekly changes in CDS spreads using contemporaneous explanatory variables. The explanatory variables include the risk free rate, leverage and the equity volatility as credit risk variables, the bid-ask spread as liquidity variable and the term structure slope, swap spread, corporate bond spread, market return and market volatility as business cycle and market wide variables. %The definition of these explanatory variables and all other variables used in this analysis can be found in the online appendix.
\cite{annaert2013} run their analysis for 32 listed banks from the euro area in the period from 2004 to 2010. Their results show that the proposed determinants can explain CDS spreads. However, the effect and significance of these determinants is not stable over the observation period but rather display highly dynamic characteristics. \\		

We now introduce the sentiment indicator as described in \mbox{Equation~(\ref{eq:sentind})} as a new explanatory variable in the model of \cite{annaert2013}. Due to limited data availability the explanatory variable bid-ask spread from \cite{annaert2013} is not considered in our analysis. The multivariate panel regression model is then given as follows:

\begin{equation}
	cdsspread_{b,t} = \alpha_b + \sum_{k=1}^K\beta_{k}x_{b,k,t} + \theta s_{b,t} + \sum_{g=1}^G\gamma_gz_{g,t}+e_{b,t}
\end{equation}

whereby $b$ identifies the bank and $t$ the time period. Note that we use the time-varying bank-specific explanatory variables ($x_{b,k,t}$) and time-varying common explanatory variables ($z_{g,t}$) but extend the model by the time-varying bank-specific sentiment indicator $s_{bt}$. Using this regression model, we test following hypothesis:

\begin{hyp}
	The sentiment indicator is a determinant of CDS spreads.
\end{hyp}

Hence, we expect and test whether the coefficient for sentiment $\theta$ is different from zero. We expect $\theta<0$, meaning that negative sentiment $s_{b,t}<0$ is associated with increased CDS spreads and hence higher risk of the corresponding bank $b$. \\

The main interest of this analysis is however, whether the indicator constructed from news articles could serve as an early warning signal for distressed banks. Hence, rather than evaluating whether the sentiment indicator determines the current CDS spread, we are interested whether the indicator can predict future changes in risk proxies such as the CDS spread, the maximum decrease in stock price in a forward looking observation window or the equity volatility of the banks in our samples. \\

First, we assess the predictive performance of the news sentiment indicator on CDS spread. We follow \cite{cathcart2020}, which find that their news sentiment indicator has predictive power for CDS spreads on sovereign debt. They construct a global news sentiment indicator using the Thomson Reuters News Analytics (TRNA) database and focus on 25 developed and less developed countries from various geographies from 2003 to 2014. Besides finding that their sentiment indicator is a determinant of CDS spreads using panel regression techniques, \cite{cathcart2020} find that their news sentiment indicator can predict CDS spreads using panel vector autoregression methods. \\

Based on this analysis, we adjust the panel vector autoregressive model proposed by \cite{cathcart2020} to be suitable for estimating the relationship between the news sentiment of the banks in our sample on their corresponding CDS spreads. Note that our analysis has a bank specific independent variable $s_{b,t}$ for the news sentiment. Additionally, since high frequency bank-specific controls based on fundamentals or other data sources despite market data are difficult to include, we have no further bank specific controls in the regression model. However, the model includes fixed effects, which is why time invariant bank specific characteristics are controlled for. Hence, the model is given as follows:

\begin{equation} \label{eq:panelvar-cds-exogsenti}
	cdsspread_{b,t} = \alpha_b + \sum_{k=1}^K\beta_{k}x_{k,t-1} + \sum^5_{\tau=1} \delta_{\tau}cdsspread_{b,t-\tau} + \sum^5_{\tau=1}\theta_{\tau}s_{b,t-\tau} + e_{b,t}
\end{equation}

whereby $x_{k,t-1}$ are the explanatory variables stock market, volatility premium, term premium, treasury market, investment grade and high yield following \cite{cathcart2020}. The other explanatory variables from \cite{cathcart2020} are not included since they are only suitable when analysing sovereign debt and are not replaced as discussed. Following \cite{cathcart2020}, we include the CDS spread and the sentiment indicator lagged up to five periods in the model. Using this specification of the vector autoregressive model we test following hypothesis:

\begin{hyp}
	The sentiment indicator predicts CDS returns.
\end{hyp}

Hence, we expect negative coefficients for the sentiment indicator which would suggest that negative sentiment increases the risk of a bank. However, due to the in the literature already discussed over- and underreaction to news, we might observe different magnitudes and sign changes in the coefficient for sentiment, which would be explained by an over- and underreaction of the financial market participiants to news events. \\

In the model as described in \mbox{Equation~(\ref{eq:panelvar-cds-exogsenti})}, the sentiment indicator is treated as an exogenous variable. However, there is high potential for reverse causality. News content might affect the behavior of market participants which might change CDS spreads. In the same way, changes in CDS spreads might affect the news reporting about the corresponding bank. Because of this potential bidirectional causality we fit the model as described in \mbox{Equation~(\ref{eq:panelvar-cds-exogsenti})} with the sentiment indicator as a second endogenous variable while the remaining controls remain exogenous. We therefore further analyse the following hypotheses:

\begin{hyp}
	There is a bidirectional relationship between CDS spreads and the news sentiment indicator.
\end{hyp}

\begin{hyp}
	The sentiment indicator supports the theory of over- and underreaction.
\end{hyp}

We test if there is a bidirectional hypothesis by conducting a Granger Causality test as well as looking at the impulse response functions (IRFs). To test the over- and underreaction hypothesis, we look at the effect of news sentiment on CDS spreads using the corresponding IRF. \\

Since the models so far require that there are CDS traded on the banks, a lot of banks cannot be included in the sample. Hence, to include the remaining banks of our Swiss sample in the analysis, we need to use other proxies for the risk perceived in the financial markets about the corresponding banks. Since all banks are publicly traded, risk proxies constructed from the corresponding stock price allow all banks to be included in the analysis without high demands on data availability. Hence, the remaining models focus on risk proxies derived from the stock price and are fitted on the whole sample of Swiss banks as well as the European G-SIBs. \\

First, we assume that the sentiment of the news media is reflected in the stock price of the corresponding bank in the foreseeable future. As the literature has shown, the effect of news sentiment is most pronounced for media pessimism which is able to predict falling stock prices (for example as in \citealt{tetlock2007}). Additionally, from the perspective of banking supervision, we care most about increased risk. Hence, we use the maximum decrease $md$ of the stock price $S$ over the next 14 days of the corresponding bank as a risk proxy:

\begin{equation}
	md_t = \min\left\{ 0, \min_{i\in\{1,...,14\}} \frac{S_{t+i}-S_t}{S_t} \right\}
\end{equation}


 This model is implemented in daily frequency. Hence the weighted moving average of the news sentiment as described in \mbox{Equation~(\ref{eq:sentind_wma})} is used. Additionally, since $md$ is a measure for falling stock prices only, we also fit the regression model with the adjusted news sentiment indicator as defined in \mbox{Equation~(\ref{eq_sentind_wma_adj})}. We therefore fit the following panel autoregressive models:

\begin{equation} \label{eq:mdd}
	md_{b,t} = \alpha_b + \sum_{\tau=1}^5\delta_\tau md_{b,t-\tau} + \sum_{\tau=1}^5 \theta \tilde{s}_{b,t-\tau} + e_{b,t}
\end{equation}

\begin{equation} \label{eq:mdd}
	md_{b,t} = \alpha_b + \sum_{\tau=1}^5\delta_\tau md_{b,t-\tau} + \sum_{\tau=1}^5 \theta \tilde{s}_{adj,b,t-\tau} + e_{b,t}
\end{equation}

whereby $md_{b,t}$ is the $t+14$ maximum stock price decrease of the stock price of bank $b$ at time $t$. Similarly to the model described in \mbox{Equation~(\ref{eq:panelvar-cds-exogsenti})}, $md$ as the independent variable is included as lagged control variables for up to five lags in the model. To keep the model relatively simple and due to the high frequency, we do not include further controls. However, again fixed effects control for time invariant bank specific characteristics. Using this model, we check the following hypothesis:

\begin{hyp}
	The sentiment indicator predicts the maximum decrease of a banks stock price over the following 14 days.
\end{hyp}

We expect that the coefficient to be non-zero or more specifically $\theta>0$, indicating that negative sentiment increases the maximum stock decrease within the following 14 days. \\

Another risk proxy which we can directly derive from the stock price is the corresponding volatility. We assume that negative news sentiment increases risk and uncertainty about the corresponding bank. Hence, we expect that negative news sentiment is associated with higher volatility in the following periods. To analyse this relationship, we fit both a generalised autoregressive conditional heteroscedasticity (GARCH) model with covariates, GARCH-X, as well as a heterogeneous autoregressive (HAR) model with covariates. Following the implementation of the GARCH-X model as in \cite{sucarrat2021} and the HAR model as in \cite{boudt2022}, only one additional covariate can be included in the model. Hence, we have the following representation of the GARCH-X model: 

\begin{equation}
	\sigma^2_{b,t} = \alpha_b + \beta_{b,1}e^2_{t-1} + \beta_{b,2}\sigma^2_{t-1} + \theta_b\tilde{s}_{b,t-1}
\end{equation}

whereby $\sigma^2_{b,t}$ is the realised volatility, $e^2_{t-1}$ is the squared shock in $t-1$ and $\tilde{s}$ is the weighted moving average of the news sentiment indicator as described in \mbox{Equation~(\ref{eq:sentind_wma})}. Additionally, we have the following representation of the HAR model as well:

\begin{equation}
	\sigma^2_{b,t} = \alpha_b + \beta_{b,1}\sigma^2_{b,t-2:t} + \beta_{b,2}\sigma^2_{b,t-6:t} + \beta_{b,3}\sigma^2_{b,t-21:t} + \theta_b \tilde{s}_{b,t-1} + e_{b,t}
\end{equation}

whereby $\sigma^2_{b,t}$ is the realised volatility at $t$. We include the volatility of the stock price of the past 3, 7 and 22 days ($\sigma^2_{b,t-2:t}$, $\sigma^2_{b,t-6:t}$ and $\sigma^2_{b,t-21:t}$) as well as the weighted moving average of the sentiment indicator $\tilde{s}_{b,t-1}$ according to \mbox{Equation~(\ref{eq:sentind_wma})}. For both models, we test the following hypothesis:

\begin{hyp}
	The sentiment indicator predicts stock price volatility.
\end{hyp} 

We expect and test whether the coefficient for the sentiment indicator is different from zero, more specifically we expect $\theta_b<0$ which would indicate that negative sentiment increases stock price volatility. Note that both the GARCH-X model and the HAR model can only be fitted for one bank at once. Hence, we do not estimate the whole panel, rather we focus on Credit Suisse and UBS as the two biggest banks in the Swiss sample, as we expect highest media coverage and market activity for these banks. \\

As discussed in \mbox{Section~\ref{sec:topicmodelling}}, the news sentiment indicator might be of different significance when we further differentiate by the topic of the article. Following \cite{roeder2020}, we fit the regression as in \mbox{Equation~(\ref{eq:panelvar-cds-exogsenti})} using a news sentiment indicator which was constructed separately for each topic to compare whether the coefficients are significantly different between the different sentiment indicators. To fit the regression, we use the same specification as in \mbox{Equation~(\ref{eq:panelvar-cds-exogsenti})}. Hence, we test the following hypothesis:

\begin{hyp}
	The sentiment of news has different impact on the risk of a bank depending on the corresponding topic.
\end{hyp}


\section{Evaluation of Hypotheses} \label{sec:results}

% hypothesis 1:
First, according to \mbox{Hypothesis~1} we check whether the sentiment indicator is a determinant of CDS spreads. In \mbox{Table~\ref{tab:cdsdet}} we show the results of the panel regression of CDS spreads following \cite{annaert2013}, whereby the main interest is in the estimated coefficient for the sentiment. As we can see, the estimated coefficient is not statistically different from zero. Still, as expected, the coefficient is negative which would mean that negative sentiment is associated with increased risk of the corresponding bank. Noticeable is the difference in magnitude of the coefficient and its standard error between the two samples. Given the results of the panel regression following \cite{annaert2013}, we cannot confirm \mbox{Hypothesis~1} that the current news sentiment of a bank is a determinant of the corresponding CDS spread. \\

\input{tables/cdsdet.tex}

% hypothesis 2,3 and 4:
Next, by evaluating the output of the panel vector autoregressive model of CDS spreads following \cite{cathcart2020}, we check whether the sentiment indicator has predictive power on CDS spreads. \mbox{Table~\ref{tab:cdspvar}} shows the model according to \mbox{Equation~(\ref{eq:panelvar-cds-exogsenti})} fitted on the European banks in Model 1 and on the Swiss banks in Model 2. For the European sample, we do not observe any coefficients of the different lagged values of the sentiment indicator which are statistically different from zero. For the Swiss sample, the coefficient for the sentiment indicator of lag order one is statistically different from zero and of economically significant magnitude. The sentiment indicators of higher lag order are still not statistically significantly different from zero. Still, the sign and magnitude might indicate a possible correction of the strong reaction of the lag order of zero. Hence, we can only partially accept \mbox{Hypothesis~2} that the sentiment indicator has predictive power for CDS spreads. \\

\input{tables/cdspvar.tex}

To further analyse this hypothesis, we conduct a Granger causality test to assess whether the sentiment indicator helps to predict the CDS spreads and vice versa. Note that this test is conducted for one bank at a time without additional control variables. We run the Granger causality test for both Credit Suisse and UBS and report the results in \mbox{Table~\ref{tab:granger_weekly}}. The results of this test suggests that the news sentiment indicator Granger-causes the CDS spreads for Credit Suisse, for UBS only the specification with 2 lags suggests this Granger-causality. Additionally, there seems to be no Granger-causality of CDS spreads on the news sentiment. Hence, this results do not support \mbox{Hypothesis~3} of the bidirectional relationship between CDS spreads and the news sentiment indicator but rather support that there is Granger-causality of news sentiment on CDS spreads only. \\

\input{tables/granger_weekly.tex}

\mbox{Figure~\ref{fig:oirfs}} further support this conclusion. As we can see, the orthogonalized impulse response functions (IRFs) suggest that shocks in the news sentiment are followed by a reaction in the CDS spreads while a shock in the CDS spreads do not seem to be followed by a reaction in the news sentiment. More specifically, CDS spreads seem not to respond to news sentiment shocks in the same time period. In the following period, we observe a negative relationship in reaction of CDS spreads on the shock of news sentiment, indicating that positive sentiment shocks reduce the risk proxied by CDS spreads and vice versa. This negative relationship is followed by a positive relationship before the complete reversal. Hence, we can observe similar dynamics as in the results of \cite{cathcart2020} and hence reference their behavioral story of self-attribution bias. Participants in financial markets are likely to overestimate the accuracy of their own projections of the implication the news event will have on the fundamentals of the corresponding bank. Hence, they overreact to the news event. This overreaction is followed by a underreaction, triggered by the now available information about the materialised impact of the news event. This behavioral bias of market participants is theoretically discussed by \cite{daniel1998}. Additionally, and in contrast to \cite{cathcart2020}, the effect of a shock in sentiment on CDS spreads according to the IRF suggests a complete reversal within 7 weeks. This would indicate that the news sentiment does not contain pure information but rather only noise. To summarise, this results support the \mbox{Hypothesis~4} that the sentiment indicator supports the theory of over- and underraction to news events. \\

% to do: make plot by myself and wrap this tex code in R function
\begin{figure}[H]
    \centering
    \includegraphics[width=\textwidth]{images/oirfs.png}
    \caption{Orthogonalized impulse response function}
    \label{fig:oirfs}
\end{figure}


% hypothesis 5:
Having analysed the relationship between the sentiment indicator as proposed in \mbox{Equation~(\ref{eq:sentind})} and CDS spreads, we now turn to the hypothesis whether the sentiment indicator as proposed in \mbox{Equation~(\ref{eq:sentind_wma})} has predictive performance for the maximum stock price decrease over the next 14 days of the corresponding bank. For the European G-SIBs, the estimated coefficients for the sentiment indicator are not statistically different from zero, hence the results of the model according to \mbox{Equation~(\ref{eq:mdd})} are not reported. \mbox{Table~\ref{tab:mdpvar_swiss})} reports the results of this regression for the Swiss sample. While the coefficient of the unadjusted sentiment indicator is not statistically significantly different from zero, the coefficient of the adjusted sentiment indicator is. The results indicate that negative sentiment is associated with a higher drop in the stock price. Hence, these results of the Swiss Bank sample support \mbox{Hypothesis~5} that the news sentiment indicator has predictive power of the maximum drawdown. \\

\input{tables/mdpvar_swiss.tex}

% hypothesis 6:
When evaluating both models for predicting the volatility of the banks stock price, neither the GARCH-X model as reported in \mbox{Table~\ref{tab:garchx}} nor the HAR model as reported in \mbox{Table~\ref{tab:har}} resulted in estimated coefficients for the sentiment indicator which are statistically different from zero. Hence, this analysis does not support the hypothesis 6 that the sentiment indicator has predictive performance in forecasting future stock price volatility of corresponding banks. \\

\input{tables/garchx.tex}

\input{tables/har.tex}

% hypothesis 7:
\mbox{Table~\ref{tab:topic}} shows the regression output from the same regression as in \mbox{Table~\ref{tab:cdspvar}}, but now the regressions were fitted using the sentiment scores which only considered articles with the associated topic. Compared to the general sentiment indicator, the first lag of these sentiment scores are not significantly different from zero. Additionally, the lag of order three for the topic "Product News" is significant. As stated in \mbox{Hypothesis7}, these results suggest that the sentiment of news has a different impact on the risk of a bank depending on the corresponding topic. However, in contrast to \cite{roeder2020}, these results do not confirm that non-financial topic such as legal news do significantly influence CDS spreads.

\input{tables/topic.tex}


%\section{Evaluation of Predictive Performance, out of sample analysis}
%
%here i guess i will just focus on champion model \\
%
%then also look at credit suisse at crisis \\
%
%visualize pred vs actual for cs and ubs \\


\section{Discussion} \label{sec:discussion-sec5}

The results in \mbox{Section~\ref{sec:results}} only partly support the hypotheses stated in \mbox{Section~\ref{sec:hypotheses}}. All hypotheses, the result of their evaluation as well as the method used to test the hypotheses are summarised in \mbox{Table~\ref{tab:summary-results}}.

\renewcommand{\arraystretch}{1} 
\begin{table}[H]
\centering
\resizebox{\textwidth}{!}{
\begin{tabular}{llll}
\multicolumn{2}{l}{Hypotheses ($H_0$)} & $H_0$ rejected & Method used\\ \hline
\begin{tabular}[c]{@{}l@{}}1\\\phantom{}\\\phantom{}\end{tabular} & \begin{tabular}[c]{@{}l@{}}The sentiment indicator is not\\ a determinant of CDS spreads.\\\phantom{}\end{tabular} & \begin{tabular}[c]{@{}l@{}}no\\\phantom{}\\\phantom{}\end{tabular}                                                              & \begin{tabular}[c]{@{}l@{}}Panel VAR\\\phantom{}\\\phantom{}\end{tabular} \\
\begin{tabular}[c]{@{}l@{}}2\\\phantom{}\\\phantom{}\end{tabular} & \begin{tabular}[c]{@{}l@{}}The sentiment indicator does\\ not predict CDS returns.\\\phantom{}\end{tabular} & \begin{tabular}[c]{@{}l@{}}rejected on\\ subsample\\\phantom{}\end{tabular} & \begin{tabular}[c]{@{}l@{}}Panel VAR\\\phantom{}\\\phantom{}\end{tabular} \\
\begin{tabular}[c]{@{}l@{}}3\\\phantom{}\\\phantom{}\\\phantom{}\end{tabular} & \begin{tabular}[c]{@{}l@{}}There is no bidirectional relationship\\ between CDS spreads and the news\\ sentiment indicator.\\\phantom{}\end{tabular} & \begin{tabular}[c]{@{}l@{}}rejected on\\ subsample\\\phantom{}\\\phantom{}\end{tabular} & \begin{tabular}[c]{@{}l@{}}Granger-causality tests,\\ IRFs of Panel VAR\\\phantom{}\\\phantom{}\end{tabular} \\
\begin{tabular}[c]{@{}l@{}}4\\\phantom{}\\\phantom{}\\\phantom{}\end{tabular} & \begin{tabular}[c]{@{}l@{}}The sentiment indicator does not\\ support the theory of over- and\\ underreaction.\\\phantom{}\end{tabular}  & \begin{tabular}[c]{@{}l@{}}rejected on\\ subsample\\\phantom{}\\\phantom{}\end{tabular} & \begin{tabular}[c]{@{}l@{}}IRFs of Panel VAR\\\phantom{}\\\phantom{}\\\phantom{}\end{tabular} \\
\begin{tabular}[c]{@{}l@{}}5\\\phantom{}\\\phantom{}\\\phantom{}\\\phantom{}\end{tabular} & \begin{tabular}[c]{@{}l@{}}The sentiment indicator does not\\ predict the maximum decrease of\\ a banks stock price over the following\\ 14 days.\\\phantom{}\end{tabular} & \begin{tabular}[c]{@{}l@{}}rejected on\\ subsample\\\phantom{}\\\phantom{}\\\phantom{}\end{tabular} & \begin{tabular}[c]{@{}l@{}}Panel VAR\\\phantom{}\\\phantom{}\\\phantom{}\\\phantom{}\end{tabular} \\
\begin{tabular}[c]{@{}l@{}}6\\\phantom{}\\\phantom{}\end{tabular} & \begin{tabular}[c]{@{}l@{}}The sentiment indicator does not\\ predict stock price volatility.\\\phantom{}\end{tabular} & \begin{tabular}[c]{@{}l@{}}no\\\phantom{}\\\phantom{}\end{tabular}                                                              & \begin{tabular}[c]{@{}l@{}}GARCH-X, HAR\\\phantom{}\\\phantom{}\end{tabular} \\
\begin{tabular}[c]{@{}l@{}}7\\\phantom{}\\\phantom{}\end{tabular} & \begin{tabular}[c]{@{}l@{}}The sentiment of news has no different\\ impact on the risk of a bank depending\\ on the corresponding topic.\\\end{tabular} & \begin{tabular}[c]{@{}l@{}}no\\\phantom{}\\\phantom{}\end{tabular}                                                              & \begin{tabular}[c]{@{}l@{}}Panel VAR\\\phantom{}\\\phantom{}\end{tabular}                                                                          
\end{tabular}
}
\caption{Summary of Evaluation of Hypotheses}
\label{tab:summary-results}
\end{table}
\renewcommand{\arraystretch}{1.4}




Particularly striking is the difference between the results of the European G-SIB sample and the Swiss Banks sample. One difference between the samples is the observation period used for the analysis. As the results of \cite{annaert2013} have shown, estimated coefficients and their significants are not stable over time. In particular, their results suggest that the variables included in their analysis were more influential in time of crisis. The Swiss Bank sample does include the 2023 banking crisis while the European sample does not, which might be a reason for the difference in results. \\

Another big difference is the news article database the corresponding news sentiment articles are constructed upon. The news sentiment indicator constructed for the Swiss banks sample is built on much more articles which, through aggregation, might reduce the noise by giving less weight to potential articles which do not contain much impactful information about the state of the corresponding bank. Since the general public which consumes mainstream media as contained in the database used for the Swiss sample does not care as much about recent developments as the reader of news provided by specialised vendors such as Refinitv, only highly relevant information might be published in the mainstream media. Hence, news articles published from mainstream media as in the Swiss banks sample might contain information of greater importance compared to financial news on average. Note however that as the news articles from the Refinitv Eikon API are labeled with the Refinitiv Identification Code (RIC) of the corresponding bank as well as of the nature of the database itself, it was assumed for all articles that the content is relevant for the corresponding bank. Building a more sophisticated data preprocessing pipeline for the articles retrieved from the Refinitiv Eikon API might further reduce noise and hence increase the predictive performance. \\

The results in our analysis can be interpreted as explorative studies in correlation and simple causality testing. It tries to answer whether the content of financial news articles can be quantified and used as a predictive indicator in applications relevant for banking supervision or risk management. We can see that for some periods, samples and risk proxies, there might be predictive performance in the content of media articles. To build a reliable indicator, more detailed analysis regarding sample size, observation period and robustness as well as out of sample analysis is required. As we have seen in the analysis of \mbox{Hypothesis5}, there might also be the need of further adjusting the indicator by adapting the classification rule to distinguish more drastically between positive, neutral and negative signals. \\

Finally, to build an early warning indicator, we would need to implement a decision rule on when the corresponding bank should be classified as risky enough to be further monitored by the supervisor. If adverse events of regular severity are occurring and the financial markets are reacting accordingly, there should not be a signal for the supervisor. For example, if quarterly results are not as good as expected but still not as worse to threaten the financial stability of the bank, which could be reflected in the news coverage as well as the risk proxies, a financially sound bank should not become financially distressed. Thus, a threshold of the corresponding risk proxy could be defined. Crossing this threshold could then be the decision rule to initiate a more in-depth monitoring process. There might also be a threshold which could be implemented which decides whether a predicted increase in the corresponding risk proxy could be a normal market reaction or a sign of increased financial distress. \\

The sentiment indicator itself could also be constructed in a different way such that the indicator itself could be used as a decision rule. The sentiment indicator as of now does not distinguish on news on higher or lower relevance for a supervisory context. This could be done by training an LLM by providing a metric for each textual input which indicates the level of relevance. Another possibility would be to define classes of topics which are highly relevant in the supervisory context and construct the sentiment indicator only on news about these predefined categories. Additionally, the construction of the sentiment indicator could further be extended such that the textual inputs could be classified not only as negative but also in slightly negative, negative or highly negative (for example by assigning values -1, -2 and -3 accordingly). For this, it would be required to train a novel LLM which could classify the relevance and severity of negative news in a supervisory perspective. Given the BERT model used in this analysis, one could construct the sentiment indicator by multiplying the assigned value of -1, and 1 for positive news with the probability of the predicted class which is provided in the output of the model. This should give less weight to less clearly positive or negative articles and vice versa. \\

Furthermore, the analysis could be extended by predicting spreads between the risk proxies of the different banks. For example, one could define the bank with the lowest risk proxy or the average risk proxy as a benchmark. Then, it could be analysed whether the spread of the banks to the benchmark can be predicted using a sentiment indicator. One could also analyse whether the spread of the sentiment score of a single bank to the general sentiment towards the banking sector as a whole has predictive power. Rather than just predicting the risk proxy, this might reveal if the risk of a specific bank deteriorates abnormally in comparison to its peers. \\

%Talk here also about PCA? \\

\textcolor{red}{
\textbf{das kommt von sectoin 2, muss noch hier eingebettet werden!}\\
Note that a lot of measures used for nowcasting the current riskiness of a bank are derived from market data, since other measures such as accounting based metrics are only available in a relatively low frequency. However, there are banks under supervision which are not publicly traded, making these measures unavailable. Additionally, even for publicly traded banks, metrics such as Credit Default Swaps or metrics derived from the option market might still not be available since the corresponding financial product is not traded at all or does not have a sufficiently high trading volume. Hence, nowcasting the risk of the banks concerned might be additionally challenging. The suggested news sentiment indicator does not have this requirement. Additionally, the finance literature has shown that news sentiment can be a leading indicator in the financial markets and therefore might signal increased risk earlier than market derived measures. Although the sentiment indicator does not require that the bank is publicly traded, there should be sufficient media coverage of the corresponding bank. Else, the construction of a high frequency might not be possible or could be driven by a few articles with extreme sentiment. For the Swiss banking sector, this indicator could be of great value, since three of the four systemically important banks are not publicly traded.
}

\cleardoublepage
